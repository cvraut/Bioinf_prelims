\documentclass[11pt]{article}

% use the free version of the Arial font
\usepackage[scaled]{uarial}
\renewcommand*\familydefault{\sfdefault} 
\usepackage[T1]{fontenc}

% set margins to be 0.5 in
\usepackage[legalpaper, portrait, margin=0.5in]{geometry}

% set all font sizes in this document to be size 11pt
\renewcommand{\tiny}{\normalsize}
\renewcommand{\footnotesize}{\normalsize}
\renewcommand{\small}{\normalsize}
\renewcommand{\large}{\normalsize}
\renewcommand{\Large}{\normalsize}
\renewcommand{\LARGE}{\normalsize}
\renewcommand{\huge}{\normalsize}
\renewcommand{\Huge}{\normalsize}

\begin{document}
\section*{Preliminary Exam (Specific Aims)}

Metabolic syndrome (MetS) is a combination of several health risk factors, including abdominal obesity, insulin resistance, hypertension, and dyslipidemia, that together increase the risk for cardiovascular disease, type 2 diabetes, stroke, and other chronic conditions \cite{pmid29480368}. Up to 1 in 3 Americans currently satisfy the MetS conditions making it a significant public health challenge \cite{TODO}. The underlying mechanisms of metabolic syndrome are complex and not fully understood, but are thought to involve a genetic component with MetS having an estimated heritability of 20-30\% \cite{TODO}. There is a need for further research to determine the underlying causes of MetS and to develop effective strategies for prevention and treatment.

Recently, with the integration of Next-Generation Sequencing and longer read technology, we have been able to perform Whole Genome (WGS) and Whole Exome Sequencing (WES) on Biobanks of individuals resulting in high-quality data on low-frequency genetic variants. Previous studies have identified hundreds of associations between common genetic variants and MetS using Genome Wide Association Studies (GWAS) on databases of up to 300,000 individuals. However, little is known about the effects of rare variants on MetS and no large scale (>1000 individuals) rare-variant associations have been conducted on MetS. Low-frequency genetic variants are worth studying for three reasons: they make up most of the sites of variation across the genome, they are predicted to result in significant phenotypic changes if they reside in protein-coding regions, and studying effects in protein-coding regions also provides a direct route for future functional studies and therapeutic targets. 

To understand the role genetic variants have on MetS we want to identify and study its associations with rare, protein-coding variations using data from Biobanks. Exonic variants have the potential to alter proteins which then directly affect MetS phenotypes. Therefore, studying protein-altering variants offers direct insight into the biological mechanisms of MetS. We hypothesize that using data from large Biobanks, we can detect exonic variants that result in protein-coding changes which affect MetS. This will establish a series of protein-altering variants that provide insight into factors of MetS either for improvements in genetic testing or follow-ups for functional studies to investigate the biological mechanisms underlying the associations.

\section*{Aim 1: Determine Novel Rare Variant Associations with MetS.}

\subsection*{Aim 1a: Identify Effects of rare variants on components of disease.}

\subsection*{Aim 1b: Integrate discovered signals across components to study wholistic effect on disease.}

\section*{Aim 2: Determine sequence specific mechanisms of disease progression.}

\subsection*{Aim 2a: Identify any Mendelian diseases associated with MetS significant rare variants.}

\subsection*{Aim 2b: Design CRISPR CAS9 KO guides based on exonic variants.}

\newpage

\bibliography{demo} 
\bibliographystyle{ieeetr}

\end{document}
